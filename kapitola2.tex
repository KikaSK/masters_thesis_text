\chapter{Theoretical background}
\label{chap2} % id kapitoly pre prikaz ref

\section{Implicit surfaces}
\label{sub2.1}

Implicit functions are a tool for surface representation and manipulation.
In computer graphics, they can be used for modelling of complex surfaces
using boolean operations, realistic animations, rendering and other.

Implicit functions do not define the boundary explicitly, instead
the surface is defined as a zero set of a function.
\begin{definition}
    Given a function $F: \R^3 \to \R$, one can define an implicit surface
    as a set of points that fullfil $F(x, y, z) = 0$.
\end{definition}

Some examples of implicit surfaces and their equations can be seen on
image \ref{img:1}.

\begin{figure}
    \centerline{\includegraphics[width=0.85\textwidth]{images/img1}}
    \caption[Implicit surfaces with corresponding equations]
    {Implicit surfaces with corresponding equations.}
    %id obrazku, pomocou ktoreho sa budeme na obrazok odvolavat
    \label{img:1}
\end{figure}

Unit normal vector of the implicit surface in point $(x_0, y_0, z_0)$ is
normalized gradient of the implicit function in that point.

\begin{definition}
    Gradient vector of an implicit function $F:\R^3 \to \R$ is defined as 
    $$\nabla F(x, y, z) = \bigg(\frac{\partial F(x, y, z)}{\partial x}, \frac{\partial F(x, y, z)}{\partial y}, 
    \frac{\partial F(x, y, z)}{\partial z}\bigg).$$
    If $\nabla F(x,y,z) \neq 0$, we can define the unit normal vector of $F$ as 
    a normalized gradient vector
    $$N(F(x, y, z))  = \frac{\nabla F(x, y, z)}{\| \nabla F(x, y, z) \|} \,\,\,\,\,\, for \,\,\,\,\,\,
    \nabla F(x,y,z) \neq 0.$$
    \\*
\end{definition}

Points lying on the implcit surface can be classified as regular or
singular based on the value of the gradient vector in that point.

\begin{definition}
    Point $P=(x,y,z)$ lying on the implicit surface is said to be regular,
    if $\nabla F(x, y, z) \neq 0$. On the contrary, point $P$ is said to be 
    singular, if $\nabla F(x, y, z) = 0$.
\end{definition}

Singular points can be further classified as isolated or non-isolated
based on their surroundings.

\begin{definition}
    Singular point $P$ is said to be isolated, if there exists an open ball
    $B_\varepsilon(P)$, which does not contain any other singular point.
    Singular point P is said to be non-isolated if it is not isolated.
\end{definition}

On the image \ref{img:2} we can see example of isolated and non-isolated
singularities.

\begin{figure}
    \centerline{\includegraphics[width=0.6\textwidth]{images/img2}}
    \caption[Isolated and non-isolated singularity]
    {Isolated and non-isolated singularity.}
    %id obrazku, pomocou ktoreho sa budeme na obrazok odvolavat
    \label{img:2}
\end{figure}

\subsection*{Curvature of a surface}

Curvature is a fundamental concept in differential geometry of curves and surfaces.
In case of curves, curvature is a measure of how much does the curve differ from a 
straight line. It is defined as the inverse of the radius of the osculating circle,
which is the second order approximation of the curve.

For surfaces, curvature is a measure of how much does the surface differ from a 
plane. The definition of the curvature of a surface is not as straightforward
as in the case of curves, as the curvature depends on the choice of the direction
in which we measure the curvature.

The idea of measuring the curvature of a surface has a long history in mathematics.
One of the first contributors was a mathematician Carl Friedrich Gauss, who developed
the idea of the Gaussian curvature of surfaces. In this subsection, we are drawing from
the summary presented by Tiago Novello et al.\cite{novello2021differential}.

\subsubsection*{Normal curvature of the surface}

Let $S$ be a parametric surface $$S(u, v): \R^2 \to \R^3$$ $$S(u, v) = (S_x(u,v), S_y(u,v), S_z(u,v)).$$
Let us denote the unit normal vector of the surface $S$ in the point $S(u, v)$ as $\overrightarrow{n(u, v)}$.

We define the normal curvature of the surface as a function of the location of the point
on the surface given by parameters $u$ and $v$ and the unit tangent vector in that point $\overrightarrow{u}$. 

\begin{definition}
Normal cut of a surface $S$ in the regular point $P$ in the direction of the unit tangent vector 
$\overrightarrow{u}$ is defined as an intersection of the surface $S$ and a plane
given by the vectors $\overrightarrow{u}$ and $\overrightarrow{n(u, v)}$. 
\end{definition}

The visualisation of the normal cut is shown on the image \ref{img:14}.
It is clear, that the
normal cut is a plane curve lying on the surface, we denote this normal cut as $\nu_S(u, v, \overrightarrow{u})$.

\begin{figure}
    \centerline{\includegraphics[width=0.6\textwidth]{images/img14}}
    \caption[Normal cut]
    {Normal cut of the parametric surface $S(u,v)$.}
    %id obrazku, pomocou ktoreho sa budeme na obrazok odvolavat
    \label{img:14}
\end{figure}

\begin{definition}
    Oriented normal curvature of the surface in the regular point $P$ in the direction of the unit tangent vector
    $\overrightarrow{u}$ is defined as the curvature of the normal cut $\nu_S(u, v, \overrightarrow{u})$.
    Non-oriented normal curvature is defined as an absolute value of the oriented normal curvature.
\end{definition}

\begin{definition}
    Minimal and maximal curvature in the point $P = S(u,v)$ are defined as
    $$\kappa_{min}(u,v) = \min_{\overrightarrow{u} \in T_P(u, v)} \nu_S(u, v, \overrightarrow{u}),$$
    $$\kappa_{max}(u,v) = \max_{\overrightarrow{u} \in T_P(u, v)} \nu_S(u, v, \overrightarrow{u}),$$
    where $T_S(u, v)$ is a tangent plane of the surface $S$ in the point $P$.

    Minimal and maximal curvature are called principal curvatures. 
\end{definition}

\begin{definition}
    Gaussian curvature is defined as a product of principal curvatures:
    $$\kappa_G(u, v) = \kappa_{min}(u,v) \kappa_{max}(u,v).$$
\end{definition}

Gaussian curvature describes the shape of the suface in the local neighborhood of the point.
The points where Gaussian curvature is positive are called eliptic points.
The points where Gaussian curvature is negative are called hyperbolic points.
The points where only one of $\kappa_{min}, \kappa_{max}$ is zero are called parabolic and
the points where both $\kappa_{min}$ and $\kappa_{max}$ are zero are called planar.
The shape of the surface in the local neigborhoods of the points is as follows:
\begin{itemize}
    \item {elliptic points $\longrightarrow$ sufrace is curved like a sphere,}
    \item {hyperbolic points $\longrightarrow$ surface is curved like a saddle,}
    \item {parabolic points $\longrightarrow$ surface is curved like a parabolic cylinder,}
    \item {planar points $\longrightarrow$ surface is flat.}

\end{itemize}
Gaussian curvature is an
instrinsic propery, which means that it is also independent of the placement of the surface in the space.

\begin{definition}
    Mean curvature is defined as an arithmetic mean of principal curvatures:
    $$\kappa_M(u, v) = \frac{\kappa_{min}(u,v) + \kappa_{max}(u,v)}{2}.$$
\end{definition}

Minimal, maximal, Gaussian and mean curvature are visualized on the image \ref{img:16}.

\begin{figure}
    \centerline{\includegraphics[width=0.8\textwidth]{images/img16}}
    \caption[Visualisation of the curvature of the double-torus]
    {Visualisation of the curvature of the double-torus.}
    %id obrazku, pomocou ktoreho sa budeme na obrazok odvolavat
    \label{img:16}
\end{figure}

\subsection*{Curvature formulas for implicit surface}

A version of curvature formulas for implicit surfaces appeared in \cite{spivak1975comprehensive}
and were reformulated, summarized and prooved by Ron Goldman \cite{goldman2005curvature}.
In this subsection we point out these formulas.

Let $F:\R^3 \to R$ be an implicit function which defines surface by the equation $F(x, y, z) = 0$. 
Let us denote $F_t = \frac{\partial F}{\partial t}$ and $F_{ts} = \frac{\partial^2 F}{\partial t \partial s}$.
Hessian matrix - the matrix of second derivatives is defined as 
$$
H(F) = 
\begin{pmatrix}
    F_{xx} & F_{xy} & F_{xz}\\
    F_{yx} & F_{yy} & F_{yz}\\
    F_{zx} & F_{zy} & F_{zz}
\end{pmatrix},
$$
and the adjoint of the Hessian is defined as
$$
H^*(F) = 
\begin{pmatrix}
    F_{yy} F_{zz} - F_{yz} F_{zy} & F_{yz} F_{zx} - F_{yx} F_{zz} & F_{yx} F_{zy} - F_{yy} F_{zx}\\
    F_{xz} F_{zy} - F_{xy} F_{zz} & F_{xx} F_{zz} - F_{xz} F_{zx} & F_{xy} F_{zx} - F_{xx} F_{zy}\\
    F_{xy} F_{yz} - F_{yx} F_{zy} & F_{yx} F_{xz} - F_{xx} F_{yz} & F_{xx} F_{yy} - F_{xy} F_{yx}
\end{pmatrix}.
$$

We can now formulate the formulas of Gaussian, mean, minimal and maximal curvature.

Gaussian curvature of the implicit surface defined by function $F$ is given by 
$$\kappa_G = \frac{\nabla F * H^*(F) * \nabla F^T}{|\nabla F|^4}.$$

Mean curvature of the implicit surface defined by function $F$ is given by 
$$\kappa_M = \frac{\nabla F * H^*(F) * \nabla F^T - | \nabla F |^2 Trace(H)}{2|\nabla F|^3}.$$

The principal curvatures $\kappa_{min}$ and $\kappa_{max}$ can be calculated from Gaussian curvature and
mean curvature as 
$$\kappa_{min}, \kappa_{max} = \kappa_M \pm \sqrt{\kappa_M^2-\kappa_G}.$$

\section{ADE singularities}
\label{sub2.2}

ADE singularities, also reffered to as du Val singularities are a specific
class of simple, isolated surface singularities.
They were classified by Arnold's
\cite{arnol1972normal} according to ADE classification
\cite{hazewinkel1977ubiquity} based on
correspondence of these singualrities to simply laced Dynkin diagrams
\cite{dynkin1947structure}.
We know infinitely many $A$ singularities -- $A_1, A_2, ...$,
infinitely many $D$ singularities -- $D_4, D_5, ...$ and three $E$
singularities -- $E_6, E_7$ and $E_8$.
ADE singularities are specified by their normal forms.
When working in complex space, each singularity has a single normal form:
\begin{itemize}
    \item $A_n$ \hspace{5mm} $F(x,y,z)=x^{n+1}+y^2+z^2$,
    \item $D_n$ \hspace{5mm} $F(x,y,z)=yx^2+y^{n-1}+z^2$,
    \item $E_6$ \hspace{5mm} $F(x,y,z)=x^3+y^4+z^2$,
    \item $E_7$ \hspace{5mm} $F(x,y,z)=x^3+xy^3+z^2$,
    \item $E_8$ \hspace{5mm} $F(x,y,z)=x^3+y^5+z^2$.
\end{itemize}

Each ADE singularity on a surface can be locally expressed by their
normal form.

In the real case, changing the signs in these equations produces different
surfaces and therefore, ADE singualrities can be further classified by their
signature.

\begin{definition}
    Let's mark real surface singularities based on their signature as follows:
    \begin{itemize}
        \item $A_{n\pm\pm}$ \hspace{5mm} $F(x,y,z)=x^{n+1}\pm y^2\pm z^2$,
              
        \item $D_{n\pm\pm}$ \hspace{5mm} $F(x,y,z)=yx^2\pm y^{n-1}\pm z^2$,
        
        \item $E_{6\pm\pm}$ \hspace{5mm} $F(x,y,z)=x^3\pm y^4\pm z^2$,
        
        \item $E_{7\pm\pm}$ \hspace{5mm} $F(x,y,z)=x^3\pm xy^3\pm z^2$,
        
        \item $E_{8\pm\pm}$ \hspace{5mm} $F(x,y,z)=x^3\pm y^5\pm z^2$.
        \end{itemize}
\end{definition}

The most common example of a surface with ADE singularity is an ordinary cone.
Given as the zero set of the function $F(x, y, z)=x^2-y^2-z^2$, cone has
a singular point $P=(0, 0, 0)$. This singular point is an example of $A_{1--}$
singularity.

\subsection*{Correspondence between $SO(3,\R)$ group and ADE singularities}
\label{subs2.2.2}
$SO(3, \R)$ is special orthogonal group over the field of real numbers 
in three dimensions. It is also called $3D$ rotation group, as it is a group
of all rotations about the origin in $R^3$.
\begin{definition}
    $SO(3, \R)$ is a group of $3\times3$ orthogonal matrices
    of real numbers with determinant $1$.
    $$SO(3, \R) = \bigg\{A\in\R^{3\times3} \hspace{1mm} | AA^T=I ,\hspace{1mm} det(A)=1\bigg\}.$$
\end{definition}

Simply laced Dynkin diagrams correspond to all finite subgroups of
$SO(3,\R)$. Finite subgroups of
$SO(3,\R)$ are the rotational symmetry groups of
\begin{itemize}
    \item pyramid with $n$ vertices (cyclic subgroup $\overline{C}_n$),
    \item double pyramid with $n$ vertices (dihedral subgroup $\overline{D}_n$),
    \item platonic solids
    \begin{itemize}
        \item tetrahedron (tetrahedral subgroup $\overline{T}$)
        \item octahedron (octahedral subgroup $\overline{O}$)
        \item icosahedron (icosahedral subgroup $\overline{I}$)
    \end{itemize}
\end{itemize}

The correspondence is as follows:
\begin{itemize}
    \item $A_n \iff \overline{C}_{n+1}$,
    \item $D_n \iff \overline{D}_{n+2}$,
    \item $E_6 \iff \overline{T}$,
    \item $E_7 \iff \overline{O}$,
    \item $E_8 \iff \overline{I}$.
\end{itemize}

The conclusion is that ADE singularities correspond to finite subgroups of
$SO(3, \R)$, which represent certain types of symmetries in $\R^3$.

\section{Non-isolated surface singularities}
\label{sub2.3}

\section{Tringulation of regular implicit srufaces}
\label{sub2.4}

In this section, we shortly describe the algorithm for triangulation of the regular
parts of implicit surfaces introduced in bachelor's thesis \cite{korecova2021triangulation}.

The algorithm creates triangles iteratively, one at a time. The new triangle is
created at the edge of the existing triangle. The new point is projected on the
surface using the Newton-Rhapson method. This method finds the root of the implicit
function lying on the line, which is in the direction of the gradient of the implicit
function. This approach is displayed on the figure \ref{img:29}.
\begin{figure}
    \centerline{\includegraphics[scale=0.5]{images/img29}}
    \caption[TODO]
    {TODO}
    %id obrazku, pomocou ktoreho sa budeme na obrazok odvolavat
    \label{img:29}
\end{figure}

After the new point is projected on the surface, conditions are checked. 
Some of these conditions are based on the Delaunay triangulation introduced
by Hilton \cite{hilton1996marching}. The Delaunay condition checks,
if some other points are in the proximity of circumcenter of the new triangle.
The conditions minimize the chances of triangle intersection.
If the new triangle intersects with some existing triangles, the algorithm
tries to connect the new triangle to the existing triangles in its proximity.

The algorithm is enriched with the possibility to triangulate adaptively
to the curvature of the surface using approach presented by Akkouche 
\cite{akkouche2001adaptive}. It can also triangulate surfaces in the
bounded volume - axis aligned bounding box, given by six numbers - minimal
and maximal value for each of the three axes.

As a part of our work, we reimplement the algorithm to be more effective by
using advanced data structures, such as the half-edge data structure 
\cite{kettner1999using} and the range tree \cite{lueker1978data}. 
\section{Data structures for triangulation algorithm}
\label{sub2.5}

\subsection*{Half-edge data structure}
Triangular mesh is given by a set of vertices, non-oriented edges and triangular faces.
There are multiple methods for mesh representation. The straightforward one -
list of vertices, edges and vertices does not provide any information about
the local surroundings of the vertices, edges and faces and therefore the searching
for incident faces or incident edges is complicated and inefficient.

In 1975, Baumgart \cite{baumgart1975polyhedron} presented a representation
using winged edges, which was further improved in 1985 by Weiler \cite{weiler1985edge}
who presented the modification called half-edge data structure.
Both of these representations are edge-based representations, each edge stores
references (pointers) to the surrounding vertices, edges and faces. One can easily extract the information of the
surrounding vertices, edges and faces.

In the winged edge representation, each edge is represented as oriented line segment 
and stores references to its vertices, both incident faces, left and right traverse. 
Each vertex stores a reference to one of the outgoing edges and each
face stores a reference to one of its incident edge. An example of such representation
is shown on the figure \ref{img:30} and tables \ref{tab:2}, \ref{tab:3}, \ref{tab:4}.

\begin{figure}
    \centerline{\includegraphics[scale=0.5]{images/img30}}
    \caption[Example of winged edge representation]
    {Example of winged edge representation.}
    %id obrazku, pomocou ktoreho sa budeme na obrazok odvolavat
    \label{img:30}
\end{figure}

\begin{table}[]\centering
    \begin{tabular}{|c|cc|cc|cc|cc|}
    \hline
    \hline
    Edge  & \multicolumn{2}{c|}{Vertex} & \multicolumn{2}{c|}{Face} & \multicolumn{2}{c|}{Left traverse} & \multicolumn{2}{c|}{Right traverse} \\ \hline
          & Initial      & Terminal     & Left        & Right       & Pred.            & Succ.           & Pred.            & Succ.            \\ \hline\hline
    $e_1$ & $V_1$        & $V_2$        & $F_1$       &             & $e_2$            & $e_3$           &                  &                  \\ \hline
    $e_2$ & $V_3$        & $V_1$        & $F_1$       &             & $e_3$            & $e_1$           &                  &                  \\ \hline
    $e_3$ & $V_2$        & $V_3$        & $F_1$       & $F_2$       & $e_1$            & $e_2$           & $e_5$            & $e_4$            \\ \hline
    $e_4$ & $V_2$        & $V_4$        & $F_2$       &             & $e_3$            & $e_5$           &                  &                  \\ \hline
    $e_5$ & $V_4$        & $V_3$        & $F_2$       &             & $e_4$            & $e_3$           &                  &                  \\ \hline\hline
    \end{tabular}
\caption{Edge table of a winged edge data structure for the figure \ref{img:30}.}
\label{tab:2}
\end{table}

\begin{table}[]
    \centering
    \begin{tabular}{|c|ccc|c|}
    \hline
    \hline
    Vertex  & \multicolumn{3}{c|}{Coordinates}          & Edge            \\ \hline
          & x            & y            & z           & Outgoing edge   \\ \hline\hline
    $V_1$ & $x_1$        & $y_1$        & $z_1$       & $e_1$           \\ \hline
    $V_2$ & $x_2$        & $y_2$        & $z_2$       & $e_4$           \\ \hline
    $V_3$ & $x_3$        & $y_3$        & $z_3$       & $e_2$           \\ \hline
    $V_4$ & $x_4$        & $y_4$        & $z_4$       & $e_5$           \\ \hline\hline
    \end{tabular}
\caption{Vertex table of a winged edge data structure for the figure \ref{img:30}.}
\label{tab:3}
\end{table}

\begin{table}[]
    \centering
    \begin{tabular}{|c|c|}
    \hline
    \hline
    Face  & Edge            \\ \hline\hline
    $F_1$ & $e_1$           \\ \hline
    $F_2$ & $e_4$           \\ \hline\hline
    \end{tabular}
\caption{Face table of a winged edge data structure for the figure \ref{img:30}.}
\label{tab:4}
\end{table}

In the half-edge representation, edges are split into two halves. Each half-edge
stores reference to its initial vertex, left face, opposite half-edge and left
traverse - predecessing and successing half-edge. A visualisation of the
half-edge is displayed on the figure \ref{img:32}. An example of half-edge
representation is shown on the figure \ref{img:31} and tables 
\ref{tab:5}, \ref{tab:6}, \ref{tab:7}.

\begin{figure}
    \centerline{\includegraphics[scale=0.5]{images/img32}}
    \caption[Visualisation of the half-edge data structure]
    {Visualisation of the half-edge data structure.}
    %id obrazku, pomocou ktoreho sa budeme na obrazok odvolavat
    \label{img:32}
\end{figure}

\begin{figure}
    \centerline{\includegraphics[scale=0.5]{images/img31}}
    \caption[Example of half-edge representation]
    {Example of half-edge representation.}
    %id obrazku, pomocou ktoreho sa budeme na obrazok odvolavat
    \label{img:31}
\end{figure}

\begin{table}[]\centering
    \begin{tabular}{|c|c|c|ccc|}
    \hline
    \hline
    Edge  & Vertex       & Face        & \multicolumn{3}{c|}{Edges}                            \\ \hline
          & Initial      & Incident    & Pred.            & Succ.           & Opposite         \\ \hline\hline
    $e_1$ & $V_1$        & $F_1$       & $e_2$            & $e_3$           &                  \\ \hline
    $e_2$ & $V_3$        & $F_1$       & $e_3$            & $e_1$           &                  \\ \hline
    $e_3$ & $V_2$        & $F_1$       & $e_1$            & $e_2$           & $e_4$            \\ \hline
    $e_4$ & $V_3$        & $F_2$       & $e_6$            & $e_5$           & $e_3$            \\ \hline
    $e_5$ & $V_2$        & $F_2$       & $e_4$            & $e_6$           &                  \\ \hline
    $e_6$ & $V_4$        & $F_2$       & $e_5$            & $e_4$           &                  \\ \hline\hline
    
    \end{tabular}
\caption{Edge table of a half-edge data structure for the figure \ref{img:31}.}
\label{tab:5}
\end{table}

\begin{table}[]
    \centering
    \begin{tabular}{|c|ccc|c|}
    \hline
    \hline
    Vertex  & \multicolumn{3}{c|}{Coordinates}          & Edge            \\ \hline
          & x            & y            & z           & Outgoing edge   \\ \hline\hline
    $V_1$ & $x_1$        & $y_1$        & $z_1$       & $e_1$           \\ \hline
    $V_2$ & $x_2$        & $y_2$        & $z_2$       & $e_3$           \\ \hline
    $V_3$ & $x_3$        & $y_3$        & $z_3$       & $e_2$           \\ \hline
    $V_4$ & $x_4$        & $y_4$        & $z_4$       & $e_6$           \\ \hline\hline
    \end{tabular}
\caption{Vertex table of a half-edge data structure for the figure \ref{img:31}.}
\label{tab:6}
\end{table}

\begin{table}[]
    \centering
    \begin{tabular}{|c|c|}
    \hline
    \hline
    Face  & Edge            \\ \hline\hline
    $F_1$ & $e_1$           \\ \hline
    $F_2$ & $e_4$           \\ \hline\hline
    \end{tabular}
\caption{Face table of a half-edge data structure for the figure \ref{img:31}.}
\label{tab:7}
\end{table}


\subsection*{Range tree}
\subsection*{Mesh structure}
Triangular mesh consists of vertices, edges and faces. We use the half-edge 
data structure for mesh representation and the range tree for time efficient
search of vertices in three dimensional interval.
The mesh structure used for maintaining and modifying the triangular mesh
consists of:
\begin{itemize}
    \setlength\itemsep{-2mm}
    \item list of vertices,
    \item list of half-edges,
    \item list of faces,
    \item set of active edges,
    \item set of checked edges,
    \item set of bounding edges,
    \item range tree of all vertices.
\end{itemize}
Vertex consists of 
\begin{itemize}
    \setlength\itemsep{-2mm}
    \item three coordinates,
    \item index of itself in the list of vertices,
    \item list of indices to all outgoing edges.
\end{itemize}
Half-edge consists of 
\begin{itemize}
    \setlength\itemsep{-2mm}
    \item six coordinates,
    \item index of itself in the list of half-edges,
    \item index of initial vertex in the list of vertices, 
    \item index of terminal vertex in the list of vertices,
    \item index of opposite edge in the list of half-edges,
    \item index of predecessing edge in the list of half-edges,
    \item index of successing edge in the list of half-edges,
    \item index of incident face in the list of faces.
\end{itemize}
Face consists of
\begin{itemize}
    \setlength\itemsep{-2mm}
    \item nine coordinates,
    \item index of incident half-edge in the list of half-edges.
\end{itemize}

\subsection*{Implementation}

\section{CSG modelling for implcit surfaces}
\label{sub2.6}
In this section we will discuss how constructive solid geometry can be used for
modelling complex implicit surfaces. First major publication regarding constructive
solid geometry was published in 1977 by Requicha and Voelcker \cite{requicha1977constructive}.
More detailed mathematical foundations were published a year later, in 1978 by
Requicha and Tilove \cite{requicha1978mathematical}.
\subsection*{Constructive solid geometry (CSG)}
Constructive solid geometry is a technique used for modelling complex geometric
objects using boolean operations on sets of points -- union, intersection and
difference. It gained populatity in 1980s as a powerful tool to create complex
shapes from sets of geometrical primitives, such as cylinders, spheres and cones.
The resulting model can be represented as a CSG tree, which contains geometrical
primitives in its leaves and boolean operations in its internal nodes 
\cite{foley1996computer}. An example of such CSG tree can be seen on figure 
\ref{img:18}.

\begin{figure}
    \centerline{\includegraphics[scale=0.5]{images/img18}}
    \caption[Example of CSG tree]
    {Example of CSG tree \cite{foley1996computer}.}
    %id obrazku, pomocou ktoreho sa budeme na obrazok odvolavat
    \label{img:18}
\end{figure}

\subsubsection*{CSG for implcit surfaces}
As the implicit surface divides space into inside $(F(x, y, z) < 0)$ and
outside $(F(x, y, z) > 0)$, one can easily combine the idea of
implicit surfaces and CSG modelling. Given two implicit surfaces represented
by implicit functions $F$ and $G$, the surface of the
intersection of the interiors can be defined by implicit fuction $H=min(F, G)$.
Similarly, the surface of the union of the interiors can be defined
by implicit function $H=max(F, G)$ and lastly, the surface of the difference of
the interiors can be defined by implicit function $H=max(F, -G)$. Two dimensional
example of this idea can be seen on figure \ref{img:19}.

\begin{figure}
    \centerline{\includegraphics[scale=0.5]{images/img19}}
    \caption[Boolean operations on implicit curves]
    {Boolean operations on implicit curves.}
    %id obrazku, pomocou ktoreho sa budeme na obrazok odvolavat
    \label{img:19}
\end{figure}

\begin{theorem}
    The minimum function $min(F, G)$ of two functions $F:\R^3 \to \R$ and
    $G:\R^3 \to \R$ is defined as $$min(F, G) = F + G - \sqrt{F^2+G^2}$$.
    The maximum function $max(F, G)$ of two functions $F:\R^3 \to \R$ and
    $G:\R^3 \to \R$ is defined as $$max(F, G) = F + G + \sqrt{F^2+G^2}$$.
\end{theorem}

These formulas allow us to model implicit surfaces which are the result of
performing finite number of operations union, intersection and difference
on arbitrary implicit functions.

An easy example is creating an implicit equation representing a half of
a sphere. Let us have an implicit function 
$F(x, y, z) = x^2+y^2+z^2-1$, which represents a unit sphere and another
implicit function $G(x, y, z) = z$ which represents the $xy$ plane.
The minimum function of these two functions
$$min(F, G) = x^2+y^2+y^2-1+z-\sqrt{(x^2+y^2+y^2-1)^2+z^2}$$ represents the
surface of a half sphere. The visualisation of these surfaces can be seen on
figure
\ref{img:20}.
\begin{figure}
    \centerline{\includegraphics[scale=0.5]{images/img20}}
    \caption[Intersection of a sphere and a plane]
    {Intersection of a sphere and a plane.}
    %id obrazku, pomocou ktoreho sa budeme na obrazok odvolavat
    \label{img:20}
\end{figure}