\chapter{Theoretical background}
\label{chap2} % id kapitoly pre prikaz ref

\section{Implicit surfaces}
\label{sub2.1}

Implicit functions are a tool for surface representation and manipulation.
In computer graphics, they can be used for modelling of complex surfaces
using boolean operations, realistic animations, rendering and other.

Implicit functions do not define the boundary explicitly, instead
the surface is defined as a zero set of a function.
\begin{definition}
    Given a function $F: \R^3 \to \R$, one can define an implicit surface
    as a set of points that fullfil $F(x, y, z) = 0$.
\end{definition}

Some examples of implicit surfaces and their equations can be seen on
image \ref{img:1}.

\begin{figure}
    \centerline{\includegraphics[width=0.85\textwidth]{images/img1}}
    \caption[Implicit surfaces with corresponding equations]
    {Implicit surfaces with corresponding equations.}
    %id obrazku, pomocou ktoreho sa budeme na obrazok odvolavat
    \label{img:1}
\end{figure}

Normal vector of the implicit surface in point $(x_0, y_0, z_0)$ is
normalized gradient of the implicit function in that point.

\begin{definition}
    Gradient vector of an implicit function $F:\R^3 \to \R$ is defined as 
    $$\nabla F(x, y, z) = \bigg(\frac{\partial F(x, y, z)}{\partial x}, \frac{\partial F(x, y, z)}{\partial y}, 
    \frac{\partial F(x, y, z)}{\partial z}\bigg).$$
    If $\nabla F(x,y,z) \neq 0$, we can define normal vector of $F$ as 
    a normalized gradient vector
    $$N(F(x, y, z))  = \frac{\nabla F(x, y, z)}{\| \nabla F(x, y, z) \|} \,\,\,\,\,\, for \,\,\,\,\,\,
    \nabla F(x,y,z) \neq 0.$$
    \\*
\end{definition}

Points lying on the implcit surface can be classified as regular or
singular based on the value of the gradient vector in that point.

\begin{definition}
    Point $P=(x,y,z)$ lying on the implicit surface is said to be regular,
    if $\nabla F(x, y, z) \neq 0$. On the contrary, point $P$ is said to be 
    singular, if $\nabla F(x, y, z) = 0$.
\end{definition}

Singular points can be further classified as isolated or non-isolated
based on their surroundings.

\begin{definition}
    Singular point $P$ is said to be isolated, if there exists an open ball
    $B_\varepsilon(P)$, which does not contain any other singular point.
    Singular point P is said to be non-isolated if it is not isolated.
\end{definition}

On the image \ref{img:2} we can see example of isolated and non-isolated
singularities.

\begin{figure}
    \centerline{\includegraphics[width=0.6\textwidth]{images/img2}}
    \caption[Isolated and non-isolated singularity]
    {Isolated and non-isolated singularity.}
    %id obrazku, pomocou ktoreho sa budeme na obrazok odvolavat
    \label{img:2}
\end{figure}

\subsection*{Curvature of a surface}

Curvature is a fundamental concept in differential geometry of curves and surfaces.
In case of curves, curvature is a measure of how much does the curve differ from a 
straight line. It is defined as the inverse of the radius of the osculating circle,
which is the second order approximation of the curve.

For surfaces, curvature is a measure of how much does the surface differ from a 
plane. The definition of the curvature of a surface is not as straightforward
as in the case of curves, as the curvature depends on the choice of the direction
in which we measure the curvature.

The idea of measuring the curvature of a surface has a long history in mathematics.
One of the first contributors was a mathematician Carl Friedrich Gauss, who developed
the idea of the Gaussian curvature of surfaces. In this subsection, we are drawing from
the summary presented by Tiago Novello et al.\cite{novello2021differential}.

\subsubsection*{Normal curvature of the surface}

Let $S$ be a parametric surface $$S(u, v): \R^2 \to \R^3$$ $$S(u, v) = (S_x(u,v), S_y(u,v), S_z(u,v)).$$
Let us denote the normal vector of the surface $S$ in the point $S(u, v)$ as $\overrightarrow{n(u, v)}$.

We will define the normal curvature of the surface as a function of the location of the point
on the surface given by parameters $u$ and $v$ and the unit tangent vector in that point $\overrightarrow{u}$. 

\begin{definition}
Normal cut of a surface $S$ in the regular point $P$ in the direction of the unit tangent vector 
$\overrightarrow{u}$ is defined as an intersection of the surface $S$ and a plane
given by the vectors $\overrightarrow{u}$ and $\overrightarrow{n(u, v)}$. 
\end{definition}

The visualisation of the normal cut is shown on the image \ref{img:14}.
It is clear, that the
normal cut is a plane curve lying on the surface, we will denote this normal cut as $\nu_S(u, v, \overrightarrow{u})$.

\begin{figure}
    \centerline{\includegraphics[width=0.6\textwidth]{images/img14}}
    \caption[Normal cut]
    {Normal cut of the parametric surface $S(u,v)$.}
    %id obrazku, pomocou ktoreho sa budeme na obrazok odvolavat
    \label{img:14}
\end{figure}

\begin{definition}
    Oriented normal curvature of the surface in the regular point $P$ in the direction of the unit tangent vector
    $\overrightarrow{u}$ is defined as the curvature of the normal cut $\nu_S(u, v, \overrightarrow{u})$.
    Non-oriented normal curvature is defined as an absolute value of the oriented normal curvature.
\end{definition}

\begin{definition}
    Minimal and maximal curvature in the point $P = S(u,v)$ are defined as
    $$\kappa_{min}(u,v) = \min_{\overrightarrow{u} \in T_P(u, v)} \nu_S(u, v, \overrightarrow{u}),$$
    $$\kappa_{max}(u,v) = \max_{\overrightarrow{u} \in T_P(u, v)} \nu_S(u, v, \overrightarrow{u}),$$
    where $T_S(u, v)$ is a tangent plane of the surface $S$ in the point $P$.

    Minimal and maximal curvature are called principal curvatures. 
\end{definition}

\begin{definition}
    Gaussian curvature is defined as a product of principal curvatures:
    $$\kappa_G(u, v) = \kappa_{min}(u,v) \kappa_{max}(u,v).$$
\end{definition}

Gaussian curvature describes the shape of the suface in the local neighborhood of the point.
The points where Gaussian curvature is positive are called eliptic points.
The points where Gaussian curvature is negative are called hyperbolic points.
The points where only one of $\kappa_{min}, \kappa_{max}$ is zero are called parabolic and
the points where both $\kappa_{min}$ and $\kappa_{max}$ are zero are called planar.
The shape of the surface in the local neigborhoods of the points is as follows:
\begin{itemize}
    \item {elliptic points $\longrightarrow$ sufrace is curved like a sphere,}
    \item {hyperbolic points $\longrightarrow$ surface is curved like a saddle,}
    \item {parabolic points $\longrightarrow$ surface is curved like a parabolic cylinder,}
    \item {planar points $\longrightarrow$ surface is flat.}

\end{itemize}
Gaussian curvature is an
instrinsic propery, which means it is independent of the placement of the surface in the space.

\begin{definition}
    Mean curvature is defined as an arithmetic mean of principal curvatures:
    $$\kappa_M(u, v) = \frac{\kappa_{min}(u,v) + \kappa_{max}(u,v)}{2}.$$
\end{definition}

Minimal, maximal, Gaussian and mean curvature are visualized on the image \ref{img:16}.

\begin{figure}
    \centerline{\includegraphics[width=0.8\textwidth]{images/img16}}
    \caption[Visualisation of curvature of the double-torus]
    {Visualisation of curvature of the double-torus.}
    %id obrazku, pomocou ktoreho sa budeme na obrazok odvolavat
    \label{img:16}
\end{figure}

\subsection*{Curvature formulas for implicit surface}

A version of curvature formulas for implicit surfaces appeared in \cite{spivak1975comprehensive}
and were reformulated, summarized and prooved by Ron Goldman \cite{goldman2005curvature}.
In this subsection we will point out these formulas.

Let $F:\R^3 \to R$ be an implicit function which defines surface by the equation $F(x, y, z) = 0$. 
Let us denote $F_t = \frac{\partial F}{\partial t}$ and $F_{ts} = \frac{\partial^2 F}{\partial t \partial s}$.
Hessian matrix - the matrix of second derivatives is defined as 
$$
H(F) = 
\begin{pmatrix}
    F_{xx} & F_{xy} & F_{xz}\\
    F_{yx} & F_{yy} & F_{yz}\\
    F_{zx} & F_{zy} & F_{zz}
\end{pmatrix},
$$
and the adjoint of the Hessian is defined as
$$
H^*(F) = 
\begin{pmatrix}
    F_{yy} F_{zz} - F_{yz} F_{zy} & F_{yz} F_{zx} - F_{yx} F_{zz} & F_{yx} F_{zy} - F_{yy} F_{zx}\\
    F_{xz} F_{zy} - F_{xy} F_{zz} & F_{xx} F_{zz} - F_{xz} F_{zx} & F_{xy} F_{zx} - F_{xx} F_{zy}\\
    F_{xy} F_{yz} - F_{yx} F_{zy} & F_{yx} F_{xz} - F_{xx} F_{yz} & F_{xx} F_{yy} - F_{xy} F_{yx}
\end{pmatrix}.
$$

We can now formulate the formulas of Gaussian, mean, minimal and maximal curvature.

Gaussian curvature of the implicit surface defined by function $F$ is given by 
$$\kappa_G = \frac{\nabla F * H^*(F) * \nabla F^T}{|\nabla F|^4}.$$

Mean curvature of the implicit surface defined by function $F$ is given by 
$$\kappa_M = \frac{\nabla F * H^*(F) * \nabla F^T - | \nabla F |^2 Trace(H)}{2|\nabla F|^3}.$$

The principal curvatures $\kappa_{min}$ and $\kappa_{max}$ can be calculated from Gaussian curvature and
mean curvature as 
$$\kappa_{min}, \kappa_{max} = \kappa_M \pm \sqrt{\kappa_M^2-\kappa_G}.$$

\section{ADE singularities}
\label{sub2.2}

ADE singularities, also reffered to as du Val singularities are a specific
class of simple, isolated surface singularities. They were first TODO.

\subsection*{ADE classification and simply laced Dynkin diagrams}
\label{subs2.2.1}

\begin{definition} \cite{humphreys2012introduction}
    A vector space $L$ over field $F$, with an operation $L \times L \to L$,
    denoted $(x, y) = [xy]$ and called the bracket or commutator of x and y,
    is called Lie algebra over $F$ if the following axioms are satisfied:
    \begin{itemize}
        \item The bracket operation is bilinear.
        \item $[xx] = 0$ for all $x$ in $L$.
        \item $[x[yz]]+[y[zx]]+[z[xy]] = 0$ for all $x, y, z \in L$.
    \end{itemize}

    Simple Lie algebra is non-abelian Lie algebra, which contains no
    nonzero proper ideals.

    Semisimple Lie algebra is a direct sum of simple Lie algebras.
\end{definition}

There is a one-to-one Correspondence between Lie algebras and Lie groups.

Dynkin diagrams are graphs which classify semisimple Lie algebras (or
equivalently semisimple Lie groups).
Simply laced Dynkin diagrams are undirected diagrams with
no multiple edges. Lie algebras which correspond to simply laced
Dynkin diagrams are called simply laced Lie algebras.

ADE in ADE singularities reffers to ADE classification, which is used when
some objects have a pattern that corresponds to simply laced Dynkin diagrams.

Simple Lie algebras over algebraically closed field 
(and their corresponding Lie groups) are
classified based on their Dynkin diagrams as
\begin{itemize}
\item $A_n$ \hspace{3mm}  $n>=1$,
\item $B_n$ \hspace{3mm}  $n>=2$,
\item $C_n$ \hspace{3mm}  $n>=3$,
\item $D_n$ \hspace{3mm}  $n>=4$,
\item $E_6, E_7, E_8, F_4, G_2$.
\end{itemize}
The corresponding Dynkin diagrams can be seen on image \ref{img:3}.

\begin{figure}
    \centerline{\includegraphics[width=0.3\textwidth]{images/img3}}
    \caption[Finite Dynkin diagrams]
    {Finite Dynkin diagrams\cite{wikidynkindiagram}.}
    %id obrazku, pomocou ktoreho sa budeme na obrazok odvolavat
    \label{img:3}
\end{figure}

Simply laced Dynkin diagrams are simple Dynkin diagrams with no directed
and no multiple edges. $A_n, D_n, E_6, E_7$ and $E_8$ are therefore all
simply laced Dynkin diagrams.

ADE singularities are in correspondence with simply laced Dynkin
diagrams, each ADE singularity has its corresponding simply laced Dynkin
diagram and equivalently, each simply laced Dynkin diagram corresponds
to an ADE singularity. These singularities are denoted based on their
corresponding Dynkin diagram.

The ADE surface singualrities were classified by Arnold's
\cite{arnol1972normal} and they are specified by their normal forms.
When working in complex space, each singularity has a single normal form:
\begin{itemize}
    \item $A_n$ \hspace{5mm} $F(x,y,z)=x^{n+1}+y^2+z^2$,
    \item $D_n$ \hspace{5mm} $F(x,y,z)=yx^2+y^{n-1}+z^2$,
    \item $E_6$ \hspace{5mm} $F(x,y,z)=x^3+y^4+z^2$,
    \item $E_7$ \hspace{5mm} $F(x,y,z)=x^3+xy^3+z^2$,
    \item $E_8$ \hspace{5mm} $F(x,y,z)=x^3+y^5+z^2$.
\end{itemize}

Each ADE singularity on a surface can be locally expressed by their
normal form.

In the real case, changing the signs in these equations produces different
surfaces and therefore, ADE singualrities can be further classified by their
signature.

\begin{definition}
    Let's mark real surface singularities based on their signature as follows:
    \begin{itemize}
        \item $A_{n\pm\pm}$ \hspace{5mm} $F(x,y,z)=x^{n+1}\pm y^2\pm z^2$,
              
        \item $D_{n\pm\pm}$ \hspace{5mm} $F(x,y,z)=yx^2\pm y^{n-1}\pm z^2$,
        
        \item $E_{6\pm\pm}$ \hspace{5mm} $F(x,y,z)=x^3\pm y^4\pm z^2$,
        
        \item $E_{7\pm\pm}$ \hspace{5mm} $F(x,y,z)=x^3\pm xy^3\pm z^2$,
        
        \item $E_{8\pm\pm}$ \hspace{5mm} $F(x,y,z)=x^3\pm y^5\pm z^2$.
        \end{itemize}
\end{definition}

The most common example of a surface with ADE singularity is an ordinary cone.
Given as the zero set of the function $F(x, y, z)=x^2-y^2-z^2$, cone has
a singular point $P=(0, 0, 0)$. This singular point is an example of $A_{1--}$
singularity.

\subsection*{Correspondence between $SO(3,\R)$ group and ADE singularities}
\label{subs2.2.2}
$SO(3, \R)$ is special orthogonal group over the field of real numbers 
in three dimensions. It is also called $3D$ rotation group, as it is a group
of all rotations about the origin in $R^3$.
\begin{definition}
    $SO(3, \R)$ is a group of $3\times3$ orthogonal matrices
    of real numbers with determinant $1$.
    $$SO(3, \R) = \bigg\{A\in\R^{3\times3} \hspace{1mm} | AA^T=I ,\hspace{1mm} det(A)=1\bigg\}.$$
\end{definition}

Simply laced Dynkin diagrams correspond to all finite subgroups of
$SO(3,\R)$. Finite subgroups of
$SO(3,\R)$ are the rotational symmetry groups of
\begin{itemize}
    \item pyramid with $n$ vertices (cyclic subgroup $\overline{C}_n$),
    \item double pyramid with $n$ vertices (dihedral subgroup $\overline{D}_n$),
    \item platonic solids
    \begin{itemize}
        \item tetrahedron (tetrahedral subgroup $\overline{T}$)
        \item octahedron (octahedral subgroup $\overline{O}$)
        \item icosahedron (icosahedral subgroup $\overline{I}$)
    \end{itemize}
\end{itemize}

These correspond to simply laced Dynkin diagrams:
\begin{itemize}
    \item $A_n \iff \overline{C}_{n+1}$,
    \item $D_n \iff \overline{D}_{n+2}$,
    \item $E_6 \iff \overline{T}$,
    \item $E_7 \iff \overline{O}$,
    \item $E_8 \iff \overline{I}$.
\end{itemize}

The conclusion is that ADE singularities correspond to finite subgroups of
$SO(3, \R)$, which represent certain types of symmetries in $\R^3$.

\section{Non-isolated translation surface singularities}
\label{sub2.3}

\section{Tringulation of regular implicit srufaces}
\label{sub2.4}

\section{Data structures for triangulation algorithm}
\label{sub2.5}
