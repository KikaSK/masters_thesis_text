\chapter{Our contributions}
\label{chap3}

\section{Triangulation adaptive to the local curvature}
\label{sub3.1}

As we explained in the begining of chapeter \ref{sub2.1}, curvature of the surface
is a measure of how much the surface bends.

The triangulation of the surface should be accurate enough, but also memory efficient.
This can be achieved by creating a triangulation which is locally adaptive to the
curvature of the surface. Therefore having smaller triangles in the places where
the surface is curved and having bigger triangles where surface is flatter.

In this section we present our implementation of the triangulation adaptive
to the local curvature.

In the original algorithm, the height of the triangle which is projected
to the surface is set to the constant value $\frac{\sqrt{3}}{2}e$, where $e$ 
is the required length of the side of the triangle. To achieve the adaptivity
of the triangles size, we set the height of the triangle to depend on the curvature in
the given point, as shown in the image \ref{img:15}.

\begin{figure}
    \centerline{\includegraphics[width=1\textwidth]{images/img15}}
    \caption[Adaptive height of the new triangle]
    {Adaptive height of the new triangle.}
    %id obrazku, pomocou ktoreho sa budeme na obrazok odvolavat
    \label{img:15}
\end{figure}

To identify the curved areas, we decided to use the maximal curvature. As the minimal
and maximal curvature are both signed, we want to identify the areas depending on 
the absolute value of these curvatures.

We do not allow arbitrary height of the triangle to avoid edge-cases. We decided
to restrict the allowed height to $\frac{1}{4}\frac{\sqrt{3}}{2}e>h>4\frac{\sqrt{3}}{2}e$.
We create a variable $m-$multiplicator, depending on $\kappa_T$ and set the height of
the new triangle to $h=m\frac{\sqrt{3}}{2}e$.

As $\kappa_T$ has values in range $\langle 0, \infty \rangle$.

\begin{definition}
    Let us define triangulation curvature of the surface $S$ in the point $P=S(u, v)$ as
    $\kappa_T(u, v) = max(|\kappa_{min}|, |\kappa_{max}|).$
\end{definition}

NOT FINISHED YET

\section{Triangulation of ADE singularities}
\label{sub3.2}

\subsection*{Analysis of the geometry of ADE singularities}

ADE singularities are simple, isolated surface singularities, which can be
expressed by corresponding implicit equations.

We already know, that $A_{1--}$ singularity is locally represented as a cone.
In this section we discuss geometric structure of other ADE surface singularities.

\begin{definition} (TODO rewrite)
    Let us define branch of ADE singularity as the part of the surface,
    which is connected to the rest only by the singular point.
\end{definition}

\begin{comment}
\begin{definition}
    Let us define triangulation direction of ADE singularity $A$ with radius $r$ as 
    Let us define triangulation direction of ADE singularity $A$ with radius $r$ as 
    a direction $\overrightarrow{v}$ for which 
    $$F\bigg(A+t \overrightarrow{v} \cap S_{r}(A)\bigg) < 0,$$
    where $F$ is the implicit equation defining the singularity $A$
    and $t,r \in \R^+$.
\end{definition}
As we can see, there are infinitely many axes of each ADE singularity.
\end{comment}

For our needs, we pick one triangulation vector for each branch of
each ADE singularity. This triangulation vector is normalized vector
either in the direction
each ADE singularity. This triangulation vector is normalized vector
either in the direction
of rotation symmetry axis or an intersection of reflection symmetry planes
of the corresponding branch. If the branch has only one reflection symmetry
plane, the triangulation vector is picked to lie in the reflection
symmetry plane.
of the corresponding branch. If the branch has only one reflection symmetry
plane, the triangulation vector is picked to lie in the reflection
symmetry plane.

\begin{comment}
Moreover, when placed to the singular point, triangulation vector of a branch 
is in the same half-space as the corresponding branch.
\end{comment}

In the general case, triangulation vectors serve us
as a partial information about the orientation of a singularity with 
respect to its normal form.
In the general case, triangulation vectors serve us
as a partial information about the orientation of a singularity with 
respect to its normal form.

\subsubsection*{$A_n$ singularities}

As we can see from the equations 
$F(x,y,z)=x^{n+1}\pm y^2\pm z^2$, $A_{n-+}$
singularities are just rotated $A_{n+-}$ singularities and $A_{n++}$ singularities
are a single point if $n$ is odd and reflected $A_{n--}$ singularities if $n$ is even. 
are a single point if $n$ is odd and reflected $A_{n--}$ singularities if $n$ is even. 
We therefore only discuss geometry of $A_{n--}$ and $A_{n+-}$ singularities.

$A_{n--}$ singularities are topologically equivalent to a cone if $n$ is odd, therefore
they have two branches.
If $n$ is even, they are topologically equivalent to a half cone or a plane, therefore
they have a single branch.
As $n$ gets bigger, the tip of the cone gets sharper. As $A_{n--}$ singularities
are rotationally symmetrical, we pick the direction of
axis of symmetry as triangulation vector. For a normal form, the triangulation vectors
are $(1, 0, 0)$ (and $(-1, 0, 0)$ if $n$ is odd).
are $(1, 0, 0)$ (and $(-1, 0, 0)$ if $n$ is odd).
First four $A_{n--}$ singularities can be seen on image \ref{img:4}.

\begin{figure}
    \centerline{\includegraphics[width=1\textwidth]{images/img4}}
    \caption[$A_{n--}$ singularities]
    {$A_{n--}$ singularities. \cite{singsurf}}
    %id obrazku, pomocou ktoreho sa budeme na obrazok odvolavat
    \label{img:4}
\end{figure}


$A_{n+-}$ singularities are topologically equivalent to a cone if $n$ is odd, therefore
they have two branches.
In the contrary with the previous singularities, as $n$ gets bigger, the tip
of the cone gets less sharp and flatter. Branches of these singularities have 
reflection symmetry planes $x=0$ and $y=0$, therefore we pick the vectors
$(0, 0, 1)$ and $(0, 0, -1)$ as the triangulation vectors.

If $n$ is even, $A_{n+-}$ singularities are topologically equivalent to a plane
with shape similar to hyperbolic paraboloid, therefore they have a single branch.
First four $A_{n+-}$ singularities can be seen on image \ref{img:5}.
For this case, we pick the vector $(1, 0, 0)$ as a triangulation vector as
these singularities have reflection symmetry planes $y=0$ and $z=0$.

\begin{figure}
    \centerline{\includegraphics[width=1\textwidth]{images/img5}}
    \caption[$A_{n+-}$ singularities]
    {$A_{n+-}$ singularities. \cite{singsurf}}
    %id obrazku, pomocou ktoreho sa budeme na obrazok odvolavat
    \label{img:5}
\end{figure}

\subsubsection*{$D_n$ singularities}

Given by equations $F(x,y,z)=yx^2\pm y^{n-1}\pm z^2$, we consider 8 categories.
For given sign combination and parity of n, the singularities are topologically
equivalent, with sharper(or flatter) features around the singularities for increasing
value of $n$ similar to $A_n$ singularities.

We can therefore say that $D_n$ singularities can be classified into 8 categories
locally represented by the following equations:
\begin{itemize}
    \item $D_{4++}$ \hspace{5mm} $yx^2 + y^3 + z^2$
    \item $D_{5++}$ \hspace{5mm} $yx^2 + y^4 + z^2$
    \item $D_{4+-}$ \hspace{5mm} $yx^2 + y^3 - z^2$
    \item $D_{5+-}$ \hspace{5mm} $yx^2 + y^4 - z^2$
    \item $D_{4-+}$ \hspace{5mm} $yx^2 - y^3 + z^2$
    \item $D_{5-+}$ \hspace{5mm} $yx^2 - y^4 + z^2$
    \item $D_{4--}$ \hspace{5mm} $yx^2 - y^3 - z^2$
    \item $D_{5--}$ \hspace{5mm} $yx^2 - y^4 - z^2$.
\end{itemize}

Now we look at some equivalences between these 8 categories.
$D_{4++}$ singularity is reflected $D_{4+-}$ singularity.
$D_{5++}$ singularity is reflected $D_{5--}$ singularity.
$D_{5-+}$ singularity is reflected $D_{5+-}$ singularity.
$D_{4-+}$ singularity is reflected $D_{4--}$ singularity.

We therefore only analyze geometry of $D_{n+-}$ singularities and
$D_{n--}$ singularities.

$D_{n+-}$ singularities are topologically equivalent to a plane when $n$ is
even and to a cone when $n$ is odd. Again, as $n$ gets bigger, the features
around singularities get sharper. Symmetry planes of these singularities
are $x=0$ and $z=0$, therefore we pick $(0, 1, 0)$ (and $(0, -1, 0)$ when $n$ is odd)
as trinauglation vectors. First four $D_{n+-}$ singularities can be seen on
image \ref{img:7}.

\begin{figure}
    \centerline{\includegraphics[width=1\textwidth]{images/img7}}
    \caption[$D_{n+-}$ singularities]
    {$D_{n+-}$ singularities. \cite{singsurf}}
    %id obrazku, pomocou ktoreho sa budeme na obrazok odvolavat
    \label{img:7}
\end{figure}


$D_{n--}$ singularities are topologically equivalent to a cone when $n$ is
odd and to a 3 halfcones connected in the singular point when $n$ is even.
First four $D_{n--}$ singularities can be seen on image \ref{img:8}.

\begin{figure}
    \centerline{\includegraphics[width=1\textwidth]{images/img8}}
    \caption[$D_{n--}$ singularities]
    {$D_{n--}$ singularities. \cite{singsurf}}
    %id obrazku, pomocou ktoreho sa budeme na obrazok odvolavat
    \label{img:8}
\end{figure}

Symmetry plane for all branches of these singularities is $z=0$.
the intersection of the surface and plane $z=0$ is displayed on image \ref{img:6}.

\begin{figure}
    \centerline{\includegraphics[width=0.7\textwidth]{images/img6}}
    \caption[Intersection of $D_{n--}$ singularities with plane $z=0$.]
    {Intersection of $D_{n--}$ singularities with plane $z=0$.}
    %id obrazku, pomocou ktoreho sa budeme na obrazok odvolavat
    \label{img:6}
\end{figure}

For $D_{n--}$ singularity, the intersections of the two branches where
$y \geq 0$ are bounded by curves $y=0$ and $x^2=y^{n-2}$. For given $r$,
we pick the triangulation vectors as $(r, \frac{1}{2}r^{\frac{2}{n-2}}, 0)$
and $(-r, \frac{1}{2}r^{\frac{2}{n-2}}, 0)$. The resulting vectors are
displayed on image \ref{img:9} by blue arrow. Parameter $r$ is changed based
on the length of the edge of triangulation triangle.
displayed on image \ref{img:9} by blue arrow. Parameter $r$ is changed based
on the length of the edge of triangulation triangle.

\begin{figure}
    \centerline{\includegraphics[width=0.35\textwidth]{images/img9}}
    \caption[Triangulation vectors for two branches of $D_{n--}$ singularities.]
    {Triangulation vectors for two branches of $D_{n--}$ singularities.}
    %id obrazku, pomocou ktoreho sa budeme na obrazok odvolavat
    \label{img:9}
\end{figure}

The third branch where $y\leq0$ has has another plane of symmetry $x=0$,
therefore triangulation vector for this branch is chosen as $(0, -1, 0)$.

\subsubsection*{$E_6, E_7$ and $E_8$ singularities}

Given by equations $F(x,y,z)=x^3\pm y^4\pm z^2$, $F(x,y,z)=x^3\pm xy^3\pm z^2$
and $F(x,y,z)=x^3\pm y^5\pm z^2$, we can see the following equivalences:
$E_{6++}$ singularity is reflected $E_{6--}$ singularity.
$E_{6+-}$ singularity is reflected $E_{6-+}$ singularity.
$E_{7+-}, E_{7-+}$ and $E_{7--}$ are all reflected $E_{7++}$ singularity.
$E_{8+-}, E_{8-+}$ and $E_{8--}$ are all reflected $E_{8++}$ singularity.

We only analyze geometry of $E_{6++}$, $E_{6+-}$, $E_{7++}$ and $E_{8++}$
singularities. These singularities are displayed on the image \ref{img:12}.


\begin{figure}
    \centerline{\includegraphics[width=1\textwidth]{images/img12}}
    \caption[$E_n$ singularities.]
    {$E_n$ singularities. \cite{singsurf}}
    %id obrazku, pomocou ktoreho sa budeme na obrazok odvolavat
    \label{img:12}
\end{figure}
Both $E_{6++}$ and $E_{6+-}$ are topologically equivalent to a plane, thus
they each have only one branch. The planes of symmetry of both of these 
branches are $y=0$ and $z=0$, therefore we pick $(-1, 0, 0)$ as the
triangulation vector.

$E_{7++}$ singularity is topologically equivalent to a cone, therefore it has
$E_{7++}$ singularity is topologically equivalent to a cone, therefore it has
two branches. The plane of symmetry of this singularity is $z=0$.

$E_{8++}$ singularity is also topologically equivalent to a plane, therefore
it has only one branch. This branch has only one plane of symmetry $z=0$.

We again look at the intersection of the surfaces with the plane of 
symmetry, this is displayed on image \ref{img:10}.

\begin{figure}
    \centerline{\includegraphics[width=0.6\textwidth]{images/img10}}
    \caption[Intersection of $E_{7++}$ and $E_{8++}$ singularities with 
    plane $z=0$.]
    {Intersection of $E_{7++}$ and $E_{8++}$ singularities with 
    plane $z=0$.}
    %id obrazku, pomocou ktoreho sa budeme na obrazok odvolavat
    \label{img:10}
\end{figure}

For $E_{7++}$ singularity, we pick $(-1, 0, 0)$ and 
$(\frac{1}{2}r^{\frac{3}{2}}, -r, 0)$ as triangulation vectors.
For $E_{8++}$ singularity, we pick $(-1, -1, 0)$ as a triangulation vector.
These vectors are displayed on the image \ref{img:10} as blue arrows.

\subsection*{Analytical calculation of local triangulation of some ADE singularities}
For given edge size $e$, we want to calculate the local triangulation of ADE
singularities, such that edges on the border of the local triangulation
have length $e$.
\subsubsection*{$A_{n--}$ singularities}
For $A_{n--}$ singularities, we create a disc of $6$ isosceles triangles
with vertex in the singular point. The bases of these triangles create regular
hexagon in the plane $P$ parallel to the plane $x=0$, as showed on the image
\ref{img:11}.
\ref{img:11}.
\begin{figure}
    \centerline{\includegraphics[width=0.6\textwidth]{images/img11}}
    \caption[Triangulation of $A_{n--}$ singularity.]
    {Triangulation of $A_{n--}$ singularity.}
    %id obrazku, pomocou ktoreho sa budeme na obrazok odvolavat
    \label{img:11}
\end{figure}
Given by equation $x^{n+1}-y^2-z^2=0$, we find the distance of the 
plane $P$ from the plane $x=0$ for the given length $e$ of the sides of
the hexagon.

Let $e$ be the length of the side of the hexagon, then the circumscribed
circle has radius $e$. This circle is identical with the intersection of
the surface and the plane $x=h$. The equation of the intersecting circle
is $y^2+z^2=h^{n+1}$ therefore, the radius can be also expressed as 
$r=h^{\frac{n+1}{2}}$, which emerges $h=e^{\frac{2}{n+1}}$. Knowing the
distance of the plane, one can easily calculate the length of the arms of
the triangles using Pythagorean theorem: 
$$a^2=h^2+e^2 \implies a = \sqrt{e^{\frac{4}{n+1}} + e^2}$$

\subsubsection*{$D_n$ singualrities}
Some $D_n$ singularities have branches with elliptical intersection with 
a plane parallel to the plane $y=0$. As ellipses have 2 axes of symmetry,
we create 8 triangles for these branches.

Let us have an ellipse $E$ with semi-major axis $a$ and semi-minor axis $b$.
We are going to create eight triangles with apex in the singular point. The other
points of the triangles lie on the ellipse and they have the same length
of the base.

As displayed on image \ref{img:13}, we pick the leftmpost, the rightmost, the top
and the bottom points. Then we can calculate the point $P$ on ellipse equidistant
from points $P_1$ and $P_2$.

\begin{figure}
    \centerline{\includegraphics[width=0.7\textwidth]{images/img13}}
    \caption[Equidistant points on ellipse.]
    {Equidistant points on ellipse.}
    %id obrazku, pomocou ktoreho sa budeme na obrazok odvolavat
    \label{img:13}
\end{figure}

$$\frac{1}{2}(a,b) + \frac{t}{2}(b,a) \in E \implies 
\frac{(a+tb)^2}{4a^2} + \frac{(b+ta)^2}{4b^2} = 1$$
$$4b^2(a^2+2atb+t^2b^2)+4a^2(b^2+2atb+t^2a^2)-a^2b^2=0$$
$$4(b^4+a^4)t^2+8ab(b^2+a^2)t+7a^2b^2=0$$
$$t=\frac{ab(\sqrt{3a^4+2a^2b^2+3b^4}-a^2-b^2)}{a^4+b^4}$$

$$P=\frac{1}{2}(a,b) + \frac{ab(\sqrt{3a^4+2a^2b^2+3b^4}-a^2-b^2)}{2(a^4+b^4)}(b,a)$$

TODO binary search for height

Given edge length $e$, we are not able to calculate the height in which the distance
between points $P1$ and $P$ is $e$. The visualization showing this is on the image \ref{img:17}.

\begin{figure}
    \centerline{\includegraphics[width=0.7\textwidth]{images/img17}}
    \caption[TODO.]
    {TODO.}
    %id obrazku, pomocou ktoreho sa budeme na obrazok odvolavat
    \label{img:17}
\end{figure}

We use binary search to find such height.
Given the height and the singularity class, we can calculate the semi-major axis
and semi-minor axis as 

$$D_{n+-} \hspace{3mm} :\hspace{3mm}  -hx^2+h^{n-1}-z^2 = 0 \hspace{5mm} h>0$$
$$x^2 + \frac{z^2}{h} = h^{n-2}$$
$$\frac{x^2}{h^{n-2}} + \frac{z^2}{h^{n-1}} = 1 \implies a_h=max(h^\frac{n-2}{2}, h^\frac{n-1}{2}) \land b_h=min(h^\frac{n-2}{2}, h^\frac{n-1}{2}).$$

As we can see, we get the same ellipse for $D_{n--}$ singularities:

$$D_{n--} \hspace{3mm} :\hspace{3mm}  -hx^2-h^{n-1}-z^2 = 0 \hspace{5mm} h>0$$
$$2|n \land x^2 + \frac{z^2}{h} = -h^{n-2} \implies x^2 + \frac{z^2}{h} = h^{n-2}.$$


Then 

$$P_h=\frac{1}{2}(h^\frac{n-2}{2},h^\frac{n-1}{2}) + \frac{h^\frac{2n-3}{2}(\sqrt{3h^{2n-4}+2h^{2n-3}+3h^{2n-2}}-h^{n-2}-h^{n-1})}{2(h^{2n-4}+h^{2n-2})}(h^\frac{n-1}{2},h^\frac{n-2}{2})$$

$$P_h=\frac{1}{2}(h^\frac{n-2}{2},h^\frac{n-1}{2}) + \frac{h^\frac{1}{2}(\sqrt{3+2h+3h^2}-1-h)}{2(1+h^2)}(h^\frac{n-1}{2},h^\frac{n-2}{2})$$

and we can calculate $e_h=||P_h-P_1||$.

As $e \leq a_h$, we can start the binary search on the interval
$\langle 0, a^\frac{2}{n-2}\rangle$ or $\langle 0, a^\frac{2}{n-1}\rangle$
and finish, when required precision is reached.

TODO want to try to prove that $||P_h-P_1||$ is monotone in h.

\subsubsection*{$E_6, E_7$ and $E_8$ singualrities}

\subsection*{Numerical calculation of local triangulation of ADE singularities}


\section{Triangulation of non-isolated singularities of translation surfaces}
\label{sub3.3}
