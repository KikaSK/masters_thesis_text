\chapter*{Conclusion}  % chapter* je necislovana kapitola
\addcontentsline{toc}{chapter}{Conclusion} % rucne pridanie do obsahu
\markboth{Conclusion}{Conclusion} % vyriesenie hlaviciek
The goal of this work was to develop an approach to triangulate
surfaces with certain types of singular points and singular points.
We developed an approach for creating a local mesh for ADE singularities
and singular curves arising as the curves on the intersection of two regular
surfaces.
The local mesh could then be finished by using the 
algorithm for regular parts of the implicit surfaces.

We introduced analytical solutions for the $A_{n--}$ singularities and
one branch of the $D_{n--}$ singularities. For other types of singularities, we 
introduced different numerical solutions based on the local geometry of 
the ADE singularities. Moreover, we introduced layers to improve the local 
approximation around the singular points. We optimized the numerical solution
to create triangles closer to isosceles triangles to improve the 
mesh quality.

For the singular curves given by the intersection of two regular surfaces
we created an approach for both -- closed and open singular curves.
The local mesh can be created for intersection, union or difference of the
two regular surfaces.

To achieve the triangulation adaptive to the local curvature of the surface,
we proposed a solution which determines the length of the edge of the triangle
based on the constant of detail. The constant of detail is a parameter,
which determines the relation between the size of the triangle and the curvedness
in a given point.

In the section \ref{sub4.2}, we presented a comparison with the SingSurf software 
\cite{morris2003client} in terms of the quality of the resulting mesh of ADE singularities.

Lastly, we reimplemented the algorithm for regular parts of the implicit surfaces
using a half-edge data structure and a range-tree. We implement the solution in C++
using the GiNaC library \cite{bauer2002introduction} -- open framework for symbolic 
computation within the C++ programming language.
The reimplementation sped up the runtime of the algorithm significantly when creating the meshes with a large number
of triangles.

In the future, one could develop an approach for the analytical triangulation of 
other ADE singularities. Other types of singularities could also be included in 
the solutions. For ADE singularities, we have the triangulation vectors given on 
the input. In the future, these could be calculated or approximated directly from 
the equation. For the CSG modelling, one could implement a solution for triangulation of the surfaces resulting from the intersection of multiple surfaces. In these surfaces, isolated singularities lying on the singular curves could arise. In the end,
improving the algorithm for regular parts is possible, including
creating a robust algorithm for fixing holes or postprocessing to improve the
mesh quality.

