\chapter*{Introduction} % chapter* je necislovana kapitola
\addcontentsline{toc}{chapter}{Introduction} % rucne pridanie do obsahu
\markboth{Introduction}{Introduction} % vyriesenie hlaviciek

In mathematics, implicit surface is a set of points where
the implicit function $F:\R^3 \to \R$ is zero. Thanks to
unification of constructive solid geometry modelling and 
implicit surfaces, we can now model and represent very 
complex surfaces using a single function.

In computer graphics, triangular mesh is the most common
way of surface representation and visualization. Calculating
the intersection of a ray and a triangular mesh is far 
easier task as calculating the intersection of a ray and
an implicit surface.

In this thesis we follow up on our effort in bachelor's thesis
\cite{korecova2021triangulation} to present an algorithm for
creating the triangular mesh of regular implicit surfaces.

First, we reimplement the algorithm to be more effective,
by using advanced data structures for mesh representation 
and 3D point search.

Next, we extend this algorithm to include the triangulation
of certain types of isolated singularities - ADE singularities.
We analyze the geometry of these singularities and propose 
an approach to triangulate these singularities.

Lastly, we improve the adaptive technique, which changes the
size of the triangles based on the local curvature of the surface.