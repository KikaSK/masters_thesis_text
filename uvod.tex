\chapter*{Introduction} % chapter* je necislovana kapitola
\addcontentsline{toc}{chapter}{Introduction} % rucne pridanie do obsahu
\markboth{Introduction}{Introduction} % vyriesenie hlaviciek

In mathematics, an implicit surface in $\R^3$ is a set of points where
a given function $F:\R^3 \to \R$ is zero. Due to a
unification of constructive solid geometry modelling and 
implicit surfaces, one can model and represent very 
complex surfaces using a single function.

In computer graphics, triangular mesh is the most common
way of surface representation and visualization. Calculating
the intersection of a ray and a triangular mesh is far 
easier task than calculating the intersection of a ray and
an implicit surface.

Many surfaces contain singular points where all three partial
derivatives vanish. These points may cause problems for
meshing algorithms. Some meshing algorithms, such as 
Marching Cubes \cite{lorensen1987marching} ignore these points,
which leads to insufficient surface representation.

In this thesis, we follow up on our effort in our previous work
\cite{korecova2021triangulation} to present an algorithm for
creating the triangular mesh of regular implicit surfaces.

First, we reimplement the algorithm to be more effective,
by using advanced data structures for mesh representation 
and 3D point search.

Next, we extend the algorithm to include the triangulation
of certain types of isolated singularities $-$ ADE singularities,
and non-isolated singularities $-$ curves on the intersection
of two regular surfaces.
We analyze the geometry of these singularities and propose 
an approach to create a mesh which correctly captures
the geometry of the singularities.

Lastly, we propose the adaptive technique, which changes the
size of the triangles based on the local curvature of the surface.

In the chapter \ref{chap4} we compare the quality of the meshes
with the SingSurf software \cite{morris2003client}. We also compare
the runtime of the reimplemented algorithm with the implementation
presented in our previous work \cite{korecova2021triangulation}. 